For our style guide we based ourselves upon the material design guidelines created by Google\footnote{\url{https://www.google.com/design/spec/material-design/introduction.html}}. These are design guidelines specifically made for use on all sorts of different screen sizes. It uses a flat simple design that aims to be intuitive. Implementation for these guidelines can be found in numerous of Google products and in a lot of other applications, both mobile and on the web.\\
These guidlines are also implemented in some web frameworks, we used material design lite which is implemented by google. We based this prototype upon one of the templates provided by default from this framework.\\

Below are some of the interaction as defined by our style guide:\\
\begin{itemize}

\item Standards for window interaction: Handled by the browser.
\item Standard window layout: A top bar with the title of the current page (equal to the current course in our application). This top bar should contain the main actions possible on the current page. In our prototype these are search and a action drop down menu. A menu bar containing the main pages (the different courses), information of the current user on top and if applicable a help button. This menu bar should be collapsed on devices with a small screen width.
\item Standards for buttons and menus: Buttons should have bright colour and if applicable an icon to indicate their action. If a default icon (material design provides a set of icons\footnote{\url{https://design.google.com/icons/}}) can not be found for the action, the button may contain text. For actions that add main content objects such as seen in the current screen, a so called FAB (floating action button) button should be provided in the bottom right of the screen. For toggles, icons or sliders can be used. When creating an item or for use in a form, sliders are used. When listing the action in an overview a icon should be used.\\
Menu's should be indicated with the 3 dot icon. They should drop down, have a white background and have their items as text. When all items of a menu can be illustrated with an icon, icons may be used. However this is not the case in the prototype.\\
Whenever using an icon, the usage of a tool tip (a small grey box with some helpful text that pops up when hovering over the icon) is highly recommended but not required. 
\item Standards for use of keyboard: When in a form, the tab key is used to go to the next item in the form. The enter key indicates that the user wants to commit his or hers input. These are standards widely used on the web.
\item Standards for text: In material design the roboto font is used\footnote{\url{https://www.google.com/fonts/specimen/Roboto}}. This font is used because it performs well on different screen sizes. The default size is 14 pixels for text. For titles the font size may very depending on which kind of title. The colour of the text depends on its background. For darker background (such as the sidebar) white or light grey text is used. When indicating additional information about a content object (for example the author of a question) it's recommended to use an italic font style to indicate that the text is a piece of meta information.
\item Standards for colour: Colour can be used to indicate the difference between areas. For example the difference in colour between the title of a question and the description or the difference between the main panel and the sidemenu. Items that should draw attention, such as the FAB, should have a bright colour.
\item Standards for user objects: The only relevant user object for this prototype is a question. Questions are listed in a list, each question should be initially collapsed to indicate it is not in use. When in the collapsed state, only the question title is visible. Users can however perform actions (toggle visibility, open/close) on them. When opening a question the description and potential responses become visible. Also if the question is open, the user gets the response form.
\item Standards for integration of information from other applications: Users can use the copy paste feature in their browser to copy text into the input fields.  
\end{itemize}
